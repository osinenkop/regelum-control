% !TeX document-id = {fb298762-8474-4a7b-90a0-a0c749091c0f}
%!BIB program = biber
\documentclass[12pt,twoside]{../../mitthesis}
% \documentclass[12pt,a4paper]{article}
% \documentclass{minimal}
%%%%%%%%%%%%%%%%%%%%%%%%%%%%%%%%%%%%%%%%%%%%%%%%%%%%%%%%%%%%%%%%%%%%%%%%%%%%%%%%%%%%%%%%%%%%
%%%%%%%%%%%%%%%%%%%%%%%%%%%%%%%%%%%%%%%%%%%%%%%%%%%%%%% IMPORTS
\usepackage{hyperref}
\usepackage{amsmath,amssymb,amsfonts}
\usepackage{mathtools}						% Load after amsmath and mathspec. Needed for custom delimiters
\usepackage{algorithmic}
\usepackage{graphicx}
\usepackage{textcomp}
\usepackage{subcaption}
\usepackage[shortlabels]{enumitem}
\usepackage[ruled,vlined]{algorithm2e}
\usepackage{float}
\usepackage{wrapfig}
% \usepacakge{color}						
\usepackage{xcolor}
\usepackage{setspace}
\usepackage{wasysym}
\usepackage{xspace}							% Provides \xspace 

\usepackage{lipsum}  						% Remove this, it's for the template only
\usepackage{tabularx}  						% Used in the template. Nice package, but use with care. Remove by default 
\usepackage{etoolbox}						% Provides \iftoggle
% \usepackage[mathletters]{ucs}
\usepackage[utf8]{inputenc}
%%%%%%%%%%%%%%%%%%%%%%%%%%%%%%%%%%%%%%%%%%%%%%%%%%%%%%%%%%%%%%%%%%%%%%%%%%%%%%%%%%%%%%%%%%%%
%%%%%%%%%%%%%%%%%%%%%%%%%%%%%%%%%%%%%%%%%%%%%%%%%%%%%%% ENVIRONMENTS
\newtheorem{dfn}{Definition}
\newtheorem{asm}{Assumption}
\newtheorem{prp}{Proposition}
\newtheorem{thm}{Theorem}
\newtheorem{crl}{Corollary}
\newtheorem{rem}{Remark}
\newtheorem{exm}{Example}

%%%%%%%%%%%%%%%%%%%%%%%%%%%%%%%%%%%%%%%%%%%%%%%%%%%%%%%%%%%%%%%%%%%%%%%%%%%%%%%%%%%%%%%%%%%%
%%%%%%%%%%%%%%%%%%%%%%%%%%%%%%%%%%%%%%%%%%%%%%%%%%%%%%% BIBLIOGRAPHY SETUP
\usepackage[
	backend=biber,
	maxnames=5,
	style=alphabetic,
	sorting=ydnt,
	giveninits=true,
	  doi=false,
	  url=false,
	  isbn=false	
%	citestyle=verbose-ibid
	]{biblatex}
\addbibresource{../../bib/AIDA__Nov2023.bib}
\addbibresource{../../bib/Osinenko__Novp2023.bib}

%%%%%%%%%%%%%%%%%%%%%%%%%%%%%%%%%%%%%%%%%%%%%%%%%%%%%%%%%%%%%%%%%%%%%%%%%%%%%%%%%%%%%%%%%%%%
%%%%%%%%%%%%%%%%%%%%%%%%%%%%%%%%%%%%%%%%%%%%%%%%%%%%%%% PREAMBLE 

\newtoggle{mlnotation}	% True: use ML notation, false: use control theory notation

\toggletrue{mlnotation}
\iftoggle{mlnotation}
{

\newcommand{\diff}{\mathop{}\!\mathrm{d}}								
\newcommand{\pdiff}[2]{ { \frac{\partial {#1}}{\partial {#2}} }}
\newcommand{\D}{\ensuremath{\mathcal{D}}}							
\newcommand{\eps}{{\varepsilon}}									
\newcommand{\ball}{{\mathcal B}}						
\newcommand{\clip}{{\text{clip}}}								
\newcommand{\Lip}[1]{\text{Lip}_{#1}}							
\newcommand{\sgn}{{\text{sgn}}}								
\newcommand{\diam}{{\text{diam}}}										
\newcommand{\dom}{{\text{dom}}}									
\newcommand{\ramp}{{\text{ramp}}}							
\newcommand{\co}{{\overline{\text{co}}}}							
\DeclareMathOperator*{\argmin}{\text{arg\,min}}					
\DeclareMathOperator*{\argmax}{\text{arg\,max}}					
\newcommand{\transp}{\ensuremath{^{\top}}}								
\newcommand{\inv}{\ensuremath{^{-1}}}								
\newcommand{\tovec}[1]{\ensuremath{\text{vec}}\left(#1\right)}		
\newcommand{\nrm}[1]{\left\lVert#1\right\rVert}			
\newcommand{\diag}[1]{{\text{diag}}\left(#1\right)}						
\newcommand{\abs}[1]{\left\lvert#1\right\rvert}					
\newcommand{\scal}[1]{\left\langle#1\right\rangle}				
\newcommand{\tr}[1]{{\text{tr}}\left(#1\right)}				
\newcommand{\E}[2][{}]{\mathbb E_{#1}\left[#2\right]}	
\newcommand{\Es}[2][{}]{\hat {\mathbb E}_{#1}\left[#2\right]}	
\newcommand{\PP}[1]{\mathbb P\left[#1\right]}					
\newcommand{\bigo}[1]{\mathcal O\left(#1\right)}			
\newcommand{\low}{{\text{low}}}										
\newcommand{\up}{{\text{up}}}										

\newcommand{\ra}{\rightarrow}									
\newcommand{\la}{\leftarrow}										
\newcommand{\rra}{\rightrightarrows}								
\newcommand{\ie}{\unskip, i.\,e.,\xspace}							
\newcommand{\eg}{\unskip, e.\,g.,\xspace}					
\newcommand{\sut}{\text{s.\,t.\,}}										
\newcommand{\wrt}{w.\,r.\,t. \xspace}								
\let\oldemptyset\emptyset
\let\emptyset\varnothing
\newcommand{\N}{{\mathbb{N}}}										
\newcommand{\Z}{{\mathbb{Z}}}							
\newcommand{\Q}{{\mathbb{Q}}}									
\newcommand{\R}{{\mathbb{R}}}									


\newcommand{\state}{s}												
\newcommand{\State}{S}												
\newcommand{\states}{\mathbb S}									
\newcommand{\action}{a}												
\newcommand{\Action}{A}												
\newcommand{\actions}{\mathbb A}								
\newcommand{\traj}{z}													
\newcommand{\Traj}{Z}												
\newcommand{\obs}{o}										
\newcommand{\Obs}{O}										
\newcommand{\obses}{\mathbb O}							
\newcommand{\policy}{\pi}										
\newcommand{\policies}{\Pi}										
\newcommand{\transit}{p}										
\newcommand{\reward}{r}										
\newcommand{\Reward}{R}									
\newcommand{\cost}{c}												
\newcommand{\Cost}{C}											
\newcommand{\Value}{V}										
\newcommand{\Advan}{\mathcal A}									
\newcommand{\W}{\ensuremath{\mathbb{W}}}					
\newcommand{\B}{\ensuremath{\mathbb{B}}}						
\newcommand{\G}{\ensuremath{\mathbb{G}}}						
\newcommand{\Hamilt}{\ensuremath{\mathcal{H}}}			
\newcommand{\K}{\ensuremath{\mathcal{K}}\xspace}				
\newcommand{\KL}{\ensuremath{\mathcal{KL}}\xspace}				
\newcommand{\Kinf}{\ensuremath{\mathcal{K}_{\infty}}\xspace}		
\newcommand{\KLinf}{\ensuremath{\mathcal{KL}_{\infty}}\xspace}		
\newcommand{\T}{\mathcal T}											
\newcommand{\deltau}{\Delta \tau}							
\newcommand{\dt}{\ensuremath{\mathrm{d}t}}						
\newcommand{\normpdf}{\ensuremath{\mathcal N}}						
\newcommand{\trajpdf}{\rho}										
\newcommand{\TD}{\delta}											
\newcommand{\old}{{\text{old}}}										
\newcommand{\loss}{\mathcal L}									
\newcommand{\replay}{\mathcal R}										
\newcommand{\safeset}{\ensuremath{\mathcal{S}}}						
\newcommand{\dkappa}{\kappa_{\text{dec}}}						
\DeclarePairedDelimiterX{\infdivx}[2]{(}{)}{
  #1\;\delimsize\|\;#2%
}
\newcommand{\kldiv}{d_{\text{KL}}\infdivx}							
\newcommand{\barkldiv}{\bar d_{\text{KL}}\infdivx}					
\newcommand{\spc}{{\,\,}}											
}
{
%%%%%%%%%%%%%%%%%%%%%%%%%%%%%%%%%%%%%%%%%%%%%%%%%%%%%%%%%%%%%%%%%%%%%%%%%%%%%%%%%%%%%%%%%%%%%
%%%%%%%%%%%%%%%%%%%%%%%%%%%%%%%%%%%%%%%%%%%%%%%%%%%%%%%% REQUIREMENTS
% color
% xspace (!)
% ams packages
% mathspec
% mathtools
%%%%%%%%%%%%%%%%%%%%%%%%%%%%%%%%%%%%%%%%%%%%%%%%%%%%%%%%%%%%%%%%%%%%%%%%%%%%%%%%%%%%%%%%%%%%
%%%%%%%%%%%%%%%%%%%%%%%%%%%%%%%%%%%%%%%%%%%%%%%%%%%%%%% GENERAL SYMBOLS
\newcommand{\diff}{\mathop{}\!\mathrm{d}}								% Differential
\newcommand{\pdiff}[2]{ { \frac{\partial {#1}}{\partial {#2}} } }		% Partial differentiation
\newcommand{\D}{\ensuremath{\mathcal{D}}}								% Generalized derivative
\newcommand{\eps}{{\varepsilon}}										% Epsilon
\newcommand{\ball}{{\mathcal B}}										% Ball
\newcommand{\clip}{{\text{clip}}}										% Clip function
\newcommand{\Lip}[1]{\text{Lip}_{#1}}									% Lipschitz constant of #1
\newcommand{\sgn}{{\text{sgn}}}											% Signum function
\newcommand{\diam}{{\text{diam}}}										% Diameter
\newcommand{\dom}{{\text{dom}}}											% Domain
\newcommand{\ramp}{{\text{ramp}}}										% Ramp	
\newcommand{\co}{{\overline{\text{co}}}}								% Convex closure
\DeclareMathOperator*{\argmin}{\text{arg\,min}}							% Argmin
\DeclareMathOperator*{\argmax}{\text{arg\,max}}							% Argmax
%\newcommand{\ln}{\text{ln}}												% Natural logarithm
\newcommand{\transp}{\ensuremath{^{\top}}}								% Matrix transpose
\newcommand{\inv}{\ensuremath{^{-1}}}									% Inverse
\newcommand{\tovec}[1]{\ensuremath{\text{vec}}\left(#1\right)}			% To-vector transformation
\newcommand{\nrm}[1]{\left\lVert#1\right\rVert}							% Norm
\newcommand{\diag}[1]{{\text{diag}}\left(#1\right)}						% Diagonal
\newcommand{\abs}[1]{\left\lvert#1\right\rvert}							% Absolute value
\newcommand{\scal}[1]{\left\langle#1\right\rangle}						% Scalar product
\newcommand{\tr}[1]{{\text{tr}}\left(#1\right)}							% Trace
\newcommand{\E}[2][{}]{\mathbb E_{#1}\left[#2\right]}					% Mean
\newcommand{\Es}[2][{}]{\hat {\mathbb E}_{#1}\left[#2\right]}			% Sample mean
\newcommand{\PP}[1]{\mathbb P\left[#1\right]}							% Probability
\newcommand{\bigo}[1]{\mathcal O\left(#1\right)}						% Big-o
\newcommand{\low}{{\text{low}}}											% Lower bound
\newcommand{\up}{{\text{up}}}											% Upper bound
%\newcommand\circled[1]{\tikz[baseline=(char.base)]{\node[shape=circle,draw,inner sep=1pt](char){#1};}}
%%%%%%%%%%%%%%%%%%%%%%%%%%%%%%%%%%%%%%%%%%%%%%%%%%%%%%%%%%%%%%%%%%%%%%%%%%%%%%%%%%%%%%%%%%%%%
%%%%%%%%%%%%%%%%%%%%%%%%%%%%%%%%%%%%%%%%%%%%%%%%%%%%%%%% ARROWS
\newcommand{\ra}{\rightarrow}											% Right arrow
\newcommand{\la}{\leftarrow}											% Left arrow
\newcommand{\rra}{\rightrightarrows}									% Double right arrow
%%%%%%%%%%%%%%%%%%%%%%%%%%%%%%%%%%%%%%%%%%%%%%%%%%%%%%%%%%%%%%%%%%%%%%%%%%%%%%%%%%%%%%%%%%%%
%%%%%%%%%%%%%%%%%%%%%%%%%%%%%%%%%%%%%%%%%%%%%%%%%%%%%%% ABBREVIATIONS
\newcommand{\ie}{\unskip, i.\,e.,\xspace}								% That is
\newcommand{\eg}{\unskip, e.\,g.,\xspace}								% For example
\newcommand{\sut}{\text{s.\,t.\,}}										% Such that or subject to
\newcommand{\wrt}{w.\,r.\,t. \xspace}									% With respect to
%%%%%%%%%%%%%%%%%%%%%%%%%%%%%%%%%%%%%%%%%%%%%%%%%%%%%%%%%%%%%%%%%%%%%%%%%%%%%%%%%%%%%%%%%%%%%
%%%%%%%%%%%%%%%%%%%%%%%%%%%%%%%%%%%%%%%%%%%%%%%%%%%%%%% SETS
\let\oldemptyset\emptyset
\let\emptyset\varnothing
\newcommand{\N}{{\mathbb{N}}}											% Set of natural numbers
\newcommand{\Z}{{\mathbb{Z}}}											% Set of integer numbers
\newcommand{\Q}{{\mathbb{Q}}}											% Set of rational numbers
\newcommand{\R}{{\mathbb{R}}}											% Set of real numbers
%%%%%%%%%%%%%%%%%%%%%%%%%%%%%%%%%%%%%%%%%%%%%%%%%%%%%%%%%%%%%%%%%%%%%%%%%%%%%%%%%%%%%%%%%%%%
%%%%%%%%%%%%%%%%%%%%%%%%%%%%%%%%%%%%%%%%%%%%%%%%%%%%%%%% COLORED
%\newcommand{\red}[1]{\textcolor{red}{#1}}
%\newcommand{\blue}[1]{\textcolor{blue}{#1}}
%\definecolor{dgreen}{rgb}{0.0, 0.5, 0.0}
%\newcommand{\green}[1]{\textcolor{dgreen}{#1}}
%%%%%%%%%%%%%%%%%%%%%%%%%%%%%%%%%%%%%%%%%%%%%%%%%%%%%%%%%%%%%%%%%%%%%%%%%%%%%%%%%%%%%%%%%%%%
%%%%%%%%%%%%%%%%%%%%%%%%%%%%%%%%%%%%%%%%%%%%%%%%%%%%%%% SYSTEMS AND CONTROL
\newcommand{\state}{x}													% State (as vector)
\newcommand{\State}{X}													% State (as random variable)
\newcommand{\states}{\mathbb X}											% State space
\newcommand{\action}{u}													% Action (as vector)	
\newcommand{\Action}{U}													% Action (as random variable)
\newcommand{\actions}{\mathbb U}										% Action space
\newcommand{\traj}{z}													% State-action tuple (as vector tuple)
\newcommand{\Traj}{Z}													% State-action tuple (as random variable tuple)
\newcommand{\obs}{y}													% Observation (as vector)
\newcommand{\Obs}{Y}													% Observation (as random variable)
\newcommand{\obses}{\mathbb Y}											% Observation space
\newcommand{\policy}{\rho}												% Policy (as function or distribution)
\newcommand{\policies}{\mathcal U}										% Policy space
\newcommand{\transit}{f}												% State transition map
\newcommand{\reward}{r}													% Reward (as vector)
\newcommand{\Reward}{R}													% Reward (as random varaible)
\newcommand{\cost}{c}													% Cost (as vector)
\newcommand{\Cost}{C}													% Cost (as random varaible)
\newcommand{\Value}{V}													% Value
\newcommand{\Advan}{\mathcal A}											% Advantage
\newcommand{\W}{\ensuremath{\mathbb{W}}}								% Weight space
\newcommand{\B}{\ensuremath{\mathbb{B}}}								% Basin
\newcommand{\G}{\ensuremath{\mathbb{G}}}								% Attractor (goal set)
\newcommand{\Hamilt}{\ensuremath{\mathcal{H}}}							% Hamiltonian
\newcommand{\K}{\ensuremath{\mathcal{K}}\xspace}						% Class kappa
\newcommand{\KL}{\ensuremath{\mathcal{KL}}\xspace}						% Class kappa-ell
\newcommand{\Kinf}{\ensuremath{\mathcal{K}_{\infty}}\xspace}			% Class kappa-infinity
\newcommand{\KLinf}{\ensuremath{\mathcal{KL}_{\infty}}\xspace}			% Class kappa-ell-infinity
\newcommand{\T}{\mathcal T}												% Total time
\newcommand{\deltau}{\Delta \tau}										% Time step size
\newcommand{\dt}{\ensuremath{\mathrm{d}t}}								% Time differential
\newcommand{\normpdf}{\ensuremath{\mathcal N}}							% Normal PDF
\newcommand{\trajpdf}{p}												% State-action PDF
\newcommand{\TD}{\delta}												% Temporal difference
\newcommand{\old}{{\text{old}}}											% Old (previous) index
\newcommand{\loss}{\mathcal L}											% Loss
\newcommand{\replay}{\mathcal R}										% Experience replay
\newcommand{\safeset}{\ensuremath{\mathcal{S}}}							% Safe set
\newcommand{\dkappa}{\kappa_{\text{dec}}}								% Decay kappa function

\DeclarePairedDelimiterX{\infdivx}[2]{(}{)}{%
  #1\;\delimsize\|\;#2%
}
\newcommand{\kldiv}{d_{\text{KL}}\infdivx}								% KL-divergence
\newcommand{\barkldiv}{\bar d_{\text{KL}}\infdivx}						% Average KL-divergence
%%%%%%%%%%%%%%%%%%%%%%%%%%%%%%%%%%%%%%%%%%%%%%%%%%%%%%%%%%%%%%%%%%%%%%%%%%%%%%%%%%%%%%%%%%%%%
%%%%%%%%%%%%%%%%%%%%%%%%%%%%%%%%%%%%%%%%%%%%%%%%%%%%%%%% AUX
\newcommand{\spc}{{\,\,}}												% White space to be used in logical formulas
}

\begin{document}
\section*{Common AIDA notation}

% \hypertarget{label}{target caption}
% \label{sec:common}
Hello world
\begin{table}
\caption{Table caption}
\label{table:my-table}
\begin{tabularx}{0.97\textwidth}{ |p{0.3\textwidth}|p{0.6\textwidth}| }  
\hline
$\R_{\ge 0}, \Z_{\ge 0}$ & Set of nonnegative reals, resp., integers \\ 
$\PP{\bullet}$ & Probability measure \\
$\E[f]{\bullet}$ & Expectation under distribution $f$ \\
$\nrm{\bullet}$ & Norm (context-dependent) \\
\hline
\end{tabularx}
\end{table}
Here we go \eqref{eqn_optctrl_problem}. Alrorithm \ref{alg:my-alg}. Figure \ref{fig:my-fig}. %\hyperref[table:my-table]{this table}  % \hyperref[sec:common]{this section}. % \hyperref[eqn_optctrl_problem]{this equation}. \hyperref[sec:notation]{and this section}.

\hyperlink{common-aida-notation}{Hyperlink}

\subsection*{Reinforcement learning and control notation.}
\label{sec:notation}
\begin{tabularx}{0.97\textwidth}{ |p{0.3\textwidth}|p{0.6\textwidth}| }  
\hline
$Q$ & Quality function \\ 
$r, R$ & Running objective (reward or cost) as definite, resp., random value\\
\hline
\end{tabularx}


Referencing to \ref{table:my-table} or \ref{table:my-another-table}

\begin{table}
\caption{Another table}
\label{table:my-another-table}
\begin{tabularx}{0.97\textwidth}{ |p{0.3\textwidth}|p{0.6\textwidth}| }  
\hline
$\policy$ & Policy, a law that generates actions from observations \\ 
$\state, \State$ & State, as definite, resp., random value \\ 
$\action, \Action$ & Action, as definite, resp., random value \\ 
$\obs, \Obs$ & Observation, as definite, resp., random value \\ 
$\states$ & State space \\
$\actions$ & Action space \\
$\obses$ & Observation space \\
$\policies$ & Policy space \\
$\transit$ & State dynamics law (function or probability distribution) \\ 
$\Value^\policy$ & Total objective (value or cost) of policy $\policy$ \\
$\Value^*$ & Optimum total objective (value or cost) under the optimal policy $\policy^*$ \\
$\Advan^{\policy,\policy'}$ & Advantage of policy $\policy$ relative to policy $\policy'$ \\
$\policy^\theta$ & Actor network with weights $\theta$ \\
$\hat \Value^w$ & Critic network (state-valued) with weights $w$ \\
\hline
\end{tabularx}

\end{table}
%%%%%%%%%%%%%%%%%%%%%%%%%%%%%%%%%%%%%%%%%%%%%%%%%%%%%%%%%%%%%%%%%%%%%%%%%%%%%%%%%%%%%%%%%%%%
%%%%%%%%%%%%%%%%%%%%%%%%%%%%%%%%%%%%%%%%%%%%%%%%%%%%%%%%%%%%%%%%%%%%%%%%%%%%%%%%%%%%%%%%%%%%
\section*{Introduction}

Reinforcement learning commonly addresses the following infinite-horizon optimal control and/or decision problem:
\begin{equation}
    \label{eqn_optctrl_problem}
    \begin{aligned}
    & \, \max_{\policy \in \policies} \; \Value^\policy (\state) = \\
    & \, \max_{\policy \in \policies} \; \E[\Action_t \sim \policy]{\sum\limits_{t=0}^{\infty} \gamma^t r(\State_t, \Action_t) \vert \State_0=\state},
    \end{aligned}
\end{equation}
where $\State_t \in \states$ at a time $t \in \T := \Z_{\ge0}$ is the environment's state with values in the state-space $\states$, $r$ is the reward (in general, running objective) rate, $\gamma$ is the discount factor, $\policy$ is the agent's policy of some function class $\policies$.
The running objective may be taken as a random variable $R_t$ whose probability distribution depends on the state and action.
The agent-environment loop dynamics are commonly modeled via the following Markov chain:
\begin{equation}
    \label{eqn_sysmarkov}
    \begin{aligned}
        & \State_{t+1} \sim \transit(\bullet \vert \state_t, \action_t), \spc t \in \T.
    \end{aligned}
\end{equation}
For the problem \eqref{eqn_optctrl_problem}, one can state an important recursive property of the objective optimum $\Value^*(\state)$ in the form of the Hamilton-Jacobi-Bellman (HJB) equation as follows:
\begin{equation}
    \label{eqn_hjb}
    \max_{\action \in \actions}{\{ \mathcal D^\action \Value^*(\state) + r(\state, \action) - \gamma \Value^*(\state)\}} = 0, \forall \state \in \states,
\end{equation}
where $\mathcal D^\action \Value^*(\state) := \E[S_+ \sim \transit(\bullet \vert \state, \action)]{\Value^*((\State_{+}))} - \Value^*(\state)$.
The common approaches to \eqref{eqn_optctrl_problem} are dynamic programming \cite{Bertsekas2019Reinforcementl,Lewis2009Reinforcementl} and model-predictive control \cite{Garcia1989Modelpredictiv,Borrelli2011PredictiveCont,Darby2012MPCCurrentpra,Mayne2014Modelpredictiv}.
The latter cuts the infinite horizon to some finite value $T>0$ thus considering effectively a finite-time optimal control problem.
Dynamic programming aims directly at the HJB \eqref{eqn_hjb} and solves it iteratively over a mesh in the state space $\states$ and thus belongs to the category of tabular methods.
The most significant problem with such a discretization is the curse of dimensionality, since the number of nodes in the said mesh grows exponentially with the dimension of the state space.
Evidently, dynamic programming is in general only applicable when the state-space is compact.
Furthermore, state-space discretization should be fine enough to avoid undesirable effects that may lead to a loss of stability of the agent-environment closed loop.
Reinforcement learning essentially approximates the optimum objective $\Value^*$ via a (deep) neural network.

Reference to \ref{table:my-another-table}


\begin{algorithm}
\caption{My awesome algorithm}
\label{alg:my-alg}
\begin{algorithmic}
\STATE {\bfseries Input:} $\theta_0$
\FOR {Learning iteration $i := 0 \dots \mathcal I$}
    \STATE Policy weight update
    \STATE $\theta_{i+1} \la \theta_i - \alpha_i \Es[\trajpdf^{\policy^{\theta_i}}]{ \Cost^\gamma_{0:T} \sum \limits_{t=0}^{T-1} \nabla_\theta \log \policy^{\theta_i} ( \Traj_t ) }$
    \STATE $\alpha_i > 0$, learning rate
\ENDFOR
\STATE \RETURN Near-optimal policy $\policy^{\theta_{\mathcal I}}$
\end{algorithmic}
\end{algorithm}

\begin{figure}
    \caption{My awesome figure!!!}
    \label{fig:my-fig}
    \includegraphics{img.pdf}
\end{figure}

%%%%%%%%%%%%%%%%%%%%%%%%%%%%%%%%%%%%%%%%%%%%%%%%%%%%%%%%%%%%%%%%%%%%%%%%%%%%%%%%%%%%%%%%%%%%
%%%%%%%%%%%%%%%%%%%%%%%%%%%%%%%%%%%%%%%%%%%%%%%%%%%%%%% BIBLIOGRAPHY
\printbibliography
% These commands are in case of plain bibtex
%\bibliographystyle{plain}
%\bibliography{
%bib/AIDA__Sep2023,
%bib/Osinenko__Sep2023
%}
\end{document}