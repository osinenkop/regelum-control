% !TeX document-id = {fb298762-8474-4a7b-90a0-a0c749091c0f}
%!BIB program = biber
\documentclass[12pt,twoside]{../../mitthesis}
% \documentclass[12pt,a4paper]{article}
% \documentclass{minimal}
%%%%%%%%%%%%%%%%%%%%%%%%%%%%%%%%%%%%%%%%%%%%%%%%%%%%%%%%%%%%%%%%%%%%%%%%%%%%%%%%%%%%%%%%%%%%
%%%%%%%%%%%%%%%%%%%%%%%%%%%%%%%%%%%%%%%%%%%%%%%%%%%%%%% IMPORTS
\usepackage{hyperref}
\usepackage{amsmath,amssymb,amsfonts}
\usepackage{mathtools}						% Load after amsmath and mathspec. Needed for custom delimiters
\usepackage{algorithmic}
\usepackage{graphicx}
\usepackage{textcomp}
\usepackage{subcaption}
\usepackage[shortlabels]{enumitem}
\usepackage[ruled,vlined]{algorithm2e}
\usepackage{float}
\usepackage{wrapfig}
% \usepacakge{color}						
\usepackage{xcolor}
\usepackage{setspace}
\usepackage{wasysym}
\usepackage{xspace}							% Provides \xspace 

\usepackage{lipsum}  						% Remove this, it's for the template only
\usepackage{tabularx}  						% Used in the template. Nice package, but use with care. Remove by default 
\usepackage{etoolbox}						% Provides \iftoggle
% \usepackage[mathletters]{ucs}
\usepackage[utf8]{inputenc}
%%%%%%%%%%%%%%%%%%%%%%%%%%%%%%%%%%%%%%%%%%%%%%%%%%%%%%%%%%%%%%%%%%%%%%%%%%%%%%%%%%%%%%%%%%%%
%%%%%%%%%%%%%%%%%%%%%%%%%%%%%%%%%%%%%%%%%%%%%%%%%%%%%%% ENVIRONMENTS
\newtheorem{dfn}{Definition}
\newtheorem{asm}{Assumption}
\newtheorem{prp}{Proposition}
\newtheorem{thm}{Theorem}
\newtheorem{crl}{Corollary}
\newtheorem{rem}{Remark}
\newtheorem{exm}{Example}

%%%%%%%%%%%%%%%%%%%%%%%%%%%%%%%%%%%%%%%%%%%%%%%%%%%%%%%%%%%%%%%%%%%%%%%%%%%%%%%%%%%%%%%%%%%%
%%%%%%%%%%%%%%%%%%%%%%%%%%%%%%%%%%%%%%%%%%%%%%%%%%%%%%% BIBLIOGRAPHY SETUP
\usepackage[
	backend=biber,
	maxnames=5,
	style=alphabetic,
	sorting=ydnt,
	giveninits=true,
	  doi=false,
	  url=false,
	  isbn=false	
%	citestyle=verbose-ibid
	]{biblatex}
\addbibresource{../../bib/AIDA__Nov2023.bib}
\addbibresource{../../bib/Osinenko__Novp2023.bib}

%%%%%%%%%%%%%%%%%%%%%%%%%%%%%%%%%%%%%%%%%%%%%%%%%%%%%%%%%%%%%%%%%%%%%%%%%%%%%%%%%%%%%%%%%%%%
%%%%%%%%%%%%%%%%%%%%%%%%%%%%%%%%%%%%%%%%%%%%%%%%%%%%%%% PREAMBLE 

\newtoggle{mlnotation}	% True: use ML notation, false: use control theory notation

\toggletrue{mlnotation}
\iftoggle{mlnotation}
{

\newcommand{\diff}{\mathop{}\!\mathrm{d}}								
\newcommand{\pdiff}[2]{ { \frac{\partial {#1}}{\partial {#2}} }}
\newcommand{\D}{\ensuremath{\mathcal{D}}}							
\newcommand{\eps}{{\varepsilon}}									
\newcommand{\ball}{{\mathcal B}}						
\newcommand{\clip}{{\text{clip}}}								
\newcommand{\Lip}[1]{\text{Lip}_{#1}}							
\newcommand{\sgn}{{\text{sgn}}}								
\newcommand{\diam}{{\text{diam}}}										
\newcommand{\dom}{{\text{dom}}}									
\newcommand{\ramp}{{\text{ramp}}}							
\newcommand{\co}{{\overline{\text{co}}}}							
\DeclareMathOperator*{\argmin}{\text{arg\,min}}					
\DeclareMathOperator*{\argmax}{\text{arg\,max}}					
\newcommand{\transp}{\ensuremath{^{\top}}}								
\newcommand{\inv}{\ensuremath{^{-1}}}								
\newcommand{\tovec}[1]{\ensuremath{\text{vec}}\left(#1\right)}		
\newcommand{\nrm}[1]{\left\lVert#1\right\rVert}			
\newcommand{\diag}[1]{{\text{diag}}\left(#1\right)}						
\newcommand{\abs}[1]{\left\lvert#1\right\rvert}					
\newcommand{\scal}[1]{\left\langle#1\right\rangle}				
\newcommand{\tr}[1]{{\text{tr}}\left(#1\right)}				
\newcommand{\E}[2][{}]{\mathbb E_{#1}\left[#2\right]}	
\newcommand{\Es}[2][{}]{\hat {\mathbb E}_{#1}\left[#2\right]}	
\newcommand{\PP}[1]{\mathbb P\left[#1\right]}					
\newcommand{\bigo}[1]{\mathcal O\left(#1\right)}			
\newcommand{\low}{{\text{low}}}										
\newcommand{\up}{{\text{up}}}										

\newcommand{\ra}{\rightarrow}									
\newcommand{\la}{\leftarrow}										
\newcommand{\rra}{\rightrightarrows}								
\newcommand{\ie}{\unskip, i.\,e.,\xspace}							
\newcommand{\eg}{\unskip, e.\,g.,\xspace}					
\newcommand{\sut}{\text{s.\,t.\,}}										
\newcommand{\wrt}{w.\,r.\,t. \xspace}								
\let\oldemptyset\emptyset
\let\emptyset\varnothing
\newcommand{\N}{{\mathbb{N}}}										
\newcommand{\Z}{{\mathbb{Z}}}							
\newcommand{\Q}{{\mathbb{Q}}}									
\newcommand{\R}{{\mathbb{R}}}									


\newcommand{\state}{s}												
\newcommand{\State}{S}												
\newcommand{\states}{\mathbb S}									
\newcommand{\action}{a}												
\newcommand{\Action}{A}												
\newcommand{\actions}{\mathbb A}								
\newcommand{\traj}{z}													
\newcommand{\Traj}{Z}												
\newcommand{\obs}{o}										
\newcommand{\Obs}{O}										
\newcommand{\obses}{\mathbb O}							
\newcommand{\policy}{\pi}										
\newcommand{\policies}{\Pi}										
\newcommand{\transit}{p}										
\newcommand{\reward}{r}										
\newcommand{\Reward}{R}									
\newcommand{\cost}{c}												
\newcommand{\Cost}{C}											
\newcommand{\Value}{V}										
\newcommand{\Advan}{\mathcal A}									
\newcommand{\W}{\ensuremath{\mathbb{W}}}					
\newcommand{\B}{\ensuremath{\mathbb{B}}}						
\newcommand{\G}{\ensuremath{\mathbb{G}}}						
\newcommand{\Hamilt}{\ensuremath{\mathcal{H}}}			
\newcommand{\K}{\ensuremath{\mathcal{K}}\xspace}				
\newcommand{\KL}{\ensuremath{\mathcal{KL}}\xspace}				
\newcommand{\Kinf}{\ensuremath{\mathcal{K}_{\infty}}\xspace}		
\newcommand{\KLinf}{\ensuremath{\mathcal{KL}_{\infty}}\xspace}		
\newcommand{\T}{\mathcal T}											
\newcommand{\deltau}{\Delta \tau}							
\newcommand{\dt}{\ensuremath{\mathrm{d}t}}						
\newcommand{\normpdf}{\ensuremath{\mathcal N}}						
\newcommand{\trajpdf}{\rho}										
\newcommand{\TD}{\delta}											
\newcommand{\old}{{\text{old}}}										
\newcommand{\loss}{\mathcal L}									
\newcommand{\replay}{\mathcal R}										
\newcommand{\safeset}{\ensuremath{\mathcal{S}}}						
\newcommand{\dkappa}{\kappa_{\text{dec}}}						
\DeclarePairedDelimiterX{\infdivx}[2]{(}{)}{
  #1\;\delimsize\|\;#2%
}
\newcommand{\kldiv}{d_{\text{KL}}\infdivx}							
\newcommand{\barkldiv}{\bar d_{\text{KL}}\infdivx}					
\newcommand{\spc}{{\,\,}}											
}
{
%%%%%%%%%%%%%%%%%%%%%%%%%%%%%%%%%%%%%%%%%%%%%%%%%%%%%%%%%%%%%%%%%%%%%%%%%%%%%%%%%%%%%%%%%%%%%
%%%%%%%%%%%%%%%%%%%%%%%%%%%%%%%%%%%%%%%%%%%%%%%%%%%%%%%% REQUIREMENTS
% color
% xspace (!)
% ams packages
% mathspec
% mathtools
%%%%%%%%%%%%%%%%%%%%%%%%%%%%%%%%%%%%%%%%%%%%%%%%%%%%%%%%%%%%%%%%%%%%%%%%%%%%%%%%%%%%%%%%%%%%
%%%%%%%%%%%%%%%%%%%%%%%%%%%%%%%%%%%%%%%%%%%%%%%%%%%%%%% GENERAL SYMBOLS
\newcommand{\diff}{\mathop{}\!\mathrm{d}}								% Differential
\newcommand{\pdiff}[2]{ { \frac{\partial {#1}}{\partial {#2}} } }		% Partial differentiation
\newcommand{\D}{\ensuremath{\mathcal{D}}}								% Generalized derivative
\newcommand{\eps}{{\varepsilon}}										% Epsilon
\newcommand{\ball}{{\mathcal B}}										% Ball
\newcommand{\clip}{{\text{clip}}}										% Clip function
\newcommand{\Lip}[1]{\text{Lip}_{#1}}									% Lipschitz constant of #1
\newcommand{\sgn}{{\text{sgn}}}											% Signum function
\newcommand{\diam}{{\text{diam}}}										% Diameter
\newcommand{\dom}{{\text{dom}}}											% Domain
\newcommand{\ramp}{{\text{ramp}}}										% Ramp	
\newcommand{\co}{{\overline{\text{co}}}}								% Convex closure
\DeclareMathOperator*{\argmin}{\text{arg\,min}}							% Argmin
\DeclareMathOperator*{\argmax}{\text{arg\,max}}							% Argmax
%\newcommand{\ln}{\text{ln}}												% Natural logarithm
\newcommand{\transp}{\ensuremath{^{\top}}}								% Matrix transpose
\newcommand{\inv}{\ensuremath{^{-1}}}									% Inverse
\newcommand{\tovec}[1]{\ensuremath{\text{vec}}\left(#1\right)}			% To-vector transformation
\newcommand{\nrm}[1]{\left\lVert#1\right\rVert}							% Norm
\newcommand{\diag}[1]{{\text{diag}}\left(#1\right)}						% Diagonal
\newcommand{\abs}[1]{\left\lvert#1\right\rvert}							% Absolute value
\newcommand{\scal}[1]{\left\langle#1\right\rangle}						% Scalar product
\newcommand{\tr}[1]{{\text{tr}}\left(#1\right)}							% Trace
\newcommand{\E}[2][{}]{\mathbb E_{#1}\left[#2\right]}					% Mean
\newcommand{\Es}[2][{}]{\hat {\mathbb E}_{#1}\left[#2\right]}			% Sample mean
\newcommand{\PP}[1]{\mathbb P\left[#1\right]}							% Probability
\newcommand{\bigo}[1]{\mathcal O\left(#1\right)}						% Big-o
\newcommand{\low}{{\text{low}}}											% Lower bound
\newcommand{\up}{{\text{up}}}											% Upper bound
%\newcommand\circled[1]{\tikz[baseline=(char.base)]{\node[shape=circle,draw,inner sep=1pt](char){#1};}}
%%%%%%%%%%%%%%%%%%%%%%%%%%%%%%%%%%%%%%%%%%%%%%%%%%%%%%%%%%%%%%%%%%%%%%%%%%%%%%%%%%%%%%%%%%%%%
%%%%%%%%%%%%%%%%%%%%%%%%%%%%%%%%%%%%%%%%%%%%%%%%%%%%%%%% ARROWS
\newcommand{\ra}{\rightarrow}											% Right arrow
\newcommand{\la}{\leftarrow}											% Left arrow
\newcommand{\rra}{\rightrightarrows}									% Double right arrow
%%%%%%%%%%%%%%%%%%%%%%%%%%%%%%%%%%%%%%%%%%%%%%%%%%%%%%%%%%%%%%%%%%%%%%%%%%%%%%%%%%%%%%%%%%%%
%%%%%%%%%%%%%%%%%%%%%%%%%%%%%%%%%%%%%%%%%%%%%%%%%%%%%%% ABBREVIATIONS
\newcommand{\ie}{\unskip, i.\,e.,\xspace}								% That is
\newcommand{\eg}{\unskip, e.\,g.,\xspace}								% For example
\newcommand{\sut}{\text{s.\,t.\,}}										% Such that or subject to
\newcommand{\wrt}{w.\,r.\,t. \xspace}									% With respect to
%%%%%%%%%%%%%%%%%%%%%%%%%%%%%%%%%%%%%%%%%%%%%%%%%%%%%%%%%%%%%%%%%%%%%%%%%%%%%%%%%%%%%%%%%%%%%
%%%%%%%%%%%%%%%%%%%%%%%%%%%%%%%%%%%%%%%%%%%%%%%%%%%%%%% SETS
\let\oldemptyset\emptyset
\let\emptyset\varnothing
\newcommand{\N}{{\mathbb{N}}}											% Set of natural numbers
\newcommand{\Z}{{\mathbb{Z}}}											% Set of integer numbers
\newcommand{\Q}{{\mathbb{Q}}}											% Set of rational numbers
\newcommand{\R}{{\mathbb{R}}}											% Set of real numbers
%%%%%%%%%%%%%%%%%%%%%%%%%%%%%%%%%%%%%%%%%%%%%%%%%%%%%%%%%%%%%%%%%%%%%%%%%%%%%%%%%%%%%%%%%%%%
%%%%%%%%%%%%%%%%%%%%%%%%%%%%%%%%%%%%%%%%%%%%%%%%%%%%%%%% COLORED
%\newcommand{\red}[1]{\textcolor{red}{#1}}
%\newcommand{\blue}[1]{\textcolor{blue}{#1}}
%\definecolor{dgreen}{rgb}{0.0, 0.5, 0.0}
%\newcommand{\green}[1]{\textcolor{dgreen}{#1}}
%%%%%%%%%%%%%%%%%%%%%%%%%%%%%%%%%%%%%%%%%%%%%%%%%%%%%%%%%%%%%%%%%%%%%%%%%%%%%%%%%%%%%%%%%%%%
%%%%%%%%%%%%%%%%%%%%%%%%%%%%%%%%%%%%%%%%%%%%%%%%%%%%%%% SYSTEMS AND CONTROL
\newcommand{\state}{x}													% State (as vector)
\newcommand{\State}{X}													% State (as random variable)
\newcommand{\states}{\mathbb X}											% State space
\newcommand{\action}{u}													% Action (as vector)	
\newcommand{\Action}{U}													% Action (as random variable)
\newcommand{\actions}{\mathbb U}										% Action space
\newcommand{\traj}{z}													% State-action tuple (as vector tuple)
\newcommand{\Traj}{Z}													% State-action tuple (as random variable tuple)
\newcommand{\obs}{y}													% Observation (as vector)
\newcommand{\Obs}{Y}													% Observation (as random variable)
\newcommand{\obses}{\mathbb Y}											% Observation space
\newcommand{\policy}{\rho}												% Policy (as function or distribution)
\newcommand{\policies}{\mathcal U}										% Policy space
\newcommand{\transit}{f}												% State transition map
\newcommand{\reward}{r}													% Reward (as vector)
\newcommand{\Reward}{R}													% Reward (as random varaible)
\newcommand{\cost}{c}													% Cost (as vector)
\newcommand{\Cost}{C}													% Cost (as random varaible)
\newcommand{\Value}{V}													% Value
\newcommand{\Advan}{\mathcal A}											% Advantage
\newcommand{\W}{\ensuremath{\mathbb{W}}}								% Weight space
\newcommand{\B}{\ensuremath{\mathbb{B}}}								% Basin
\newcommand{\G}{\ensuremath{\mathbb{G}}}								% Attractor (goal set)
\newcommand{\Hamilt}{\ensuremath{\mathcal{H}}}							% Hamiltonian
\newcommand{\K}{\ensuremath{\mathcal{K}}\xspace}						% Class kappa
\newcommand{\KL}{\ensuremath{\mathcal{KL}}\xspace}						% Class kappa-ell
\newcommand{\Kinf}{\ensuremath{\mathcal{K}_{\infty}}\xspace}			% Class kappa-infinity
\newcommand{\KLinf}{\ensuremath{\mathcal{KL}_{\infty}}\xspace}			% Class kappa-ell-infinity
\newcommand{\T}{\mathcal T}												% Total time
\newcommand{\deltau}{\Delta \tau}										% Time step size
\newcommand{\dt}{\ensuremath{\mathrm{d}t}}								% Time differential
\newcommand{\normpdf}{\ensuremath{\mathcal N}}							% Normal PDF
\newcommand{\trajpdf}{p}												% State-action PDF
\newcommand{\TD}{\delta}												% Temporal difference
\newcommand{\old}{{\text{old}}}											% Old (previous) index
\newcommand{\loss}{\mathcal L}											% Loss
\newcommand{\replay}{\mathcal R}										% Experience replay
\newcommand{\safeset}{\ensuremath{\mathcal{S}}}							% Safe set
\newcommand{\dkappa}{\kappa_{\text{dec}}}								% Decay kappa function

\DeclarePairedDelimiterX{\infdivx}[2]{(}{)}{%
  #1\;\delimsize\|\;#2%
}
\newcommand{\kldiv}{d_{\text{KL}}\infdivx}								% KL-divergence
\newcommand{\barkldiv}{\bar d_{\text{KL}}\infdivx}						% Average KL-divergence
%%%%%%%%%%%%%%%%%%%%%%%%%%%%%%%%%%%%%%%%%%%%%%%%%%%%%%%%%%%%%%%%%%%%%%%%%%%%%%%%%%%%%%%%%%%%%
%%%%%%%%%%%%%%%%%%%%%%%%%%%%%%%%%%%%%%%%%%%%%%%%%%%%%%%% AUX
\newcommand{\spc}{{\,\,}}												% White space to be used in logical formulas
}

\begin{document}
\section*{Problem statement}
REINFORCE is a policy gradient algorithm that solves the following problem:

\begin{equation}
    \label{eqn_reinforce_problem}
    \E[\transit, \policy^{\theta}]{\sum_{t = 0}^{T-1} \gamma ^ t \Cost(\State_{t}, \Action_{t})} \ra \min_{\theta},
\end{equation}
where:
\begin{enumerate}
    \item $\gamma \in [0, 1]$ is the \textit{discount factor} of the algorithm and is usually set to 1 in REINFORCE. 
    \item  $\transit : \states \times \actions \times \states \ \rightarrow \ \R_{\geq 0}$ is the \textit{transition probability density function} of the environment. Precicely, $\transit(\bullet \mid \state, \action)$ is the probability density of the next state conditioned on the current state $\state$ and the current action $\action$:
    $$
        \State_{t+1} \sim \transit(\bullet \mid \State_t, \Action_t), \quad \text{for all timesteps } t \in \{0, 1, \dots, T-1\} 
    $$
    It is also assumed that the initial state $\State_0$ for $t = 0$ is sampled from some distribution with an unconditional density $\transit_0 : \states \ \rightarrow \ \R_{\geq 0}$:
    $$
        \State_0 \sim \transit_0(\bullet)
    $$
    \item $\policy^{\theta} : \actions \times \states \rightarrow \R_{\geq 0}$ is the \textit{stochastic policy} of the agent that is parameterized by weights $\theta$. Precicely, $\policy^{\theta}(\bullet \mid \state)$ is the probability density of the action conditioned on the current state $\state$:
    $$
        \Action_t \sim \policy^{\theta}(\bullet \mid \State_t)
    $$
    \item $\Cost(s, a)$ denotes the cost associated with the current state $\state$ and action $\action$
\end{enumerate}

\section*{Description}
The underlying theory of the REINFORCE algorithm is captured by the relationship between the gradient of the expected total cost and the policy parameters $\theta$. 
This relationship is formally stated by the policy gradient theorem, which is mathematically represented as
\begin{equation}
    \label{eqn_reinforce_log_lh_trick}
    \begin{aligned}
    &\nabla_{\theta} \E[\transit, \policy^{\theta}]{\sum_{t = 0}^{T-1} \gamma ^ t \Cost(\State_{t}, \Action_{t})} = \\ 
    &\quad\E[\transit, \policy^{\theta}]{ \sum_{t = 0}^{T-1}\left(\sum_{t' = 0}^{T-1} \gamma ^ {t'} \Cost(\State_{t'}, \Action_{t'})\right)  \nabla_{\theta}\log \policy^{\theta}(\Action_t \mid \State_t)}
    \end{aligned}
\end{equation}
\subsection*{General idea of the algorithm}
The right-hand side of \eqref{eqn_reinforce_log_lh_trick} can be estimated using Monte Carlo sampling.
Thus, to optimize the policy's performance, we can apply gradient descent on the policy parameters. 
Starting with an initial set of weights $\theta_1$, we iteratively update the weights for $i = 1, 2, \ldots, \mathcal{I}$ learning iterations. 
Here, $\mathcal{I}$ is a predefined hyperparameter that denotes the total number of learning iterations. 
The update rule at each iteration is given by:
$$
\theta_{i + 1} = \theta_i - \alpha  \E[\transit, \policy^{\theta}]{ \sum_{t = 0}^{T-1}\left(\sum_{t' = 0}^{T-1} \gamma ^ {t'} \Cost(\State_{t'}, \Action_{t'})\right)  \nabla_{\theta}\log \policy^{\theta}(\Action_t \mid \State_t)},
$$
where $\alpha$ is the learning rate, and the term
$$
\E[\transit, \policy^{\theta}]{ \sum_{t = 0}^{T-1}\left(\sum_{t' = 0}^{T-1} \gamma ^ {t'} \Cost(\State_{t'}, \Action_{t'})\right)  \nabla_{\theta}\log \policy^{\theta}(\Action_t \mid \State_t)}
$$
is estimated by Monte Carlo sampling:
\begin{multline*}
\E[\transit, \policy^{\theta}]{ \sum_{t = 0}^{T-1}\left(\sum_{t' = 0}^{T-1} \gamma ^ {t'} \Cost(\State_{t'}, \Action_{t'})\right)  \nabla_{\theta}\log \policy^{\theta}(\Action_t \mid \State_t)}\approx \\
\frac{1}{M}\sum_{j = 1}^M \sum_{t = 0}^{T-1} \left(\sum_{t' = 0}^{T-1} \gamma ^ {t'} \Cost(\State^j_{t'}, \Action^j_{t'})\right)  \nabla_{\theta}\log \policy^{\theta}(\Action^j_t \mid \State^j_t),
\end{multline*}
where 
\begin{enumerate}
    \item $M$ is a predefined hyperparameter that denotes the total number of Monte Carlo simulations (which we call \textit{episodes}) in each learning iteration.
    \item $\Action^j_t$ is the action taken in episode $j$ at step $t$.
    \item $\State^j_t$ is the state in episode $j$ at step $t$.
\end{enumerate}
\subsection*{Baseline and "do not let the past distract you" principle}
Nevertheless, the described update approach on pratice suffers from high variance (even for large $M$), leading to inefficient gradient descent steps. 
To mitigate this issue, the equation~\eqref{eqn_reinforce_log_lh_trick} can be improved as follows:
\begin{equation}
    \label{eqn_reinforce_log_lh_trick_with_baseline_and_tail_costs}
    \begin{aligned}
    &\nabla_{\theta} \E[\transit, \policy^{\theta}]{\sum_{t = 0}^{T-1} \gamma ^ t \Cost(\State_{t}, \Action_{t})} = \\ 
    &\quad\E[\transit, \policy^{\theta}]{ \sum_{t = 0}^{T-1} \left( \sum_{t' = t}^{T-1} \gamma ^ {t'} \Cost(\State_{t'}, \Action_{t'}) - B_t \right) \nabla_{\theta}\log \policy^{\theta}(\Action_t \mid \State_t)},
    \end{aligned}
\end{equation}
where $B_t$ is the \textit{baseline}, a random variable independent of action $\Action_t$ \href{https://en.wikipedia.org/wiki/Conditional_independence#Conditional_independence_of_random_vectors}{conditioned on} $\State_t$. 

Note that in formula \eqref{eqn_reinforce_log_lh_trick_with_baseline_and_tail_costs}, as compared to \eqref{eqn_reinforce_log_lh_trick}, we incorporate two significant modifications to improve the formulation. 
Firslty, we introduce a variance reduction technique by subtracting a baseline $B_t$. 
Secondly, the full total cost $\sum_{t' = 0}^{T-1} \gamma ^ {t'} \Cost(\State_{t'}, \Action_{t'})$ is replaced by the tail total cost $\sum_{t' = t}^{T-1} \gamma ^ {t'} \Cost(\State_{t'}, \Action_{t'})$ (the approach is called \textit{do not let the past distract you}). 

The baseline function can be \textit{any} Borel measurable function of the state $\State_t$ (such as the estimated value function $B_t = \hat{\Value}(\State_t)$), or it could be any other appropriate random variable that is independent of the action $\Action_t$. 
For example, one could consider the averaged tail total costs from the prior iteration of learning. 
Specifically, let us denote by iteration $i$ the process where one obtains state-action pairs $(\State_t^j, \Action_t^j)$ for $t = 0, 1, \ldots, T - 1$ and $j = 1, 2, \ldots, M$ through Monte Carlo simulations. 
The baselines $B_t^{i+1}$ for the subsequent iteration $i+1$ can then be computed as
$$
    B_{t}^{i + 1} = \frac{1}{M}\sum_{j = 1}^M \sum_{t' = t}^{T-1} \gamma^{t'} \Cost(\State^j_{t'}, \Action^j_{t'}),
$$
Finally, we come up with the final version of REINFORCE algorithm in the subsequent section. 

\section*{REINFORCE algorithm}
\begin{algorithm}
    \caption{REINFORCE}
    \label{alg:my-alg}
    \begin{algorithmic}
    \STATE {\bfseries Input:} $\theta_1$ (initial policy weights)
    \FOR {learning iteration $i := 1 \dots \mathcal I$}
        \FOR {episode $j := 1 \dots M$}
            \STATE obtain initial state $\State_0^{j}$
            \FOR {step $t := 0 \dots T - 1$}
                \STATE sample action $\Action_t^j \sim \policy^{\theta}(\bullet \mid \State_t^{j})$
                \STATE obtain state from transition function $\State_{t+1}^j \sim \transit(\bullet \mid \State_t^j, \Action_t^j)$
            \ENDFOR
        \ENDFOR
        \STATE  Perform a policy gradient step:
        $$
            \theta_{i+1} \la \theta_i - \alpha \frac{1}{M}\sum_{j = 1}^M \sum_{t = 0}^{T-1}\sum_{t'=t}^{T-1}\left( \gamma^{t'} \Cost(\State_{t'}^j, \Action_{t'}^j) - B_{t}^i\right) \nabla_{\theta}\log \pi^{\theta}(A_t^j \mid \State_t^j)\rvert_{\theta = \theta_i},
        $$
        \STATE where the baseline $B_t^i$ may be selected from any of the types specified \hyperlink{baseline-and-do-not-let-the-past-distract-you-principle}{previously}, according to your preference.
    \ENDFOR
    \STATE \RETURN Optimal policy $\policy^{\theta_{\mathcal I}}$
    \end{algorithmic}
\end{algorithm}
\end{document}

