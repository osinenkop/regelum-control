% !TeX document-id = {fb298762-8474-4a7b-90a0-a0c749091c0f}
%!BIB program = biber
\documentclass[12pt,twoside]{../../mitthesis}
% \documentclass[12pt,a4paper]{article}
% \documentclass{minimal}
%%%%%%%%%%%%%%%%%%%%%%%%%%%%%%%%%%%%%%%%%%%%%%%%%%%%%%%%%%%%%%%%%%%%%%%%%%%%%%%%%%%%%%%%%%%%
%%%%%%%%%%%%%%%%%%%%%%%%%%%%%%%%%%%%%%%%%%%%%%%%%%%%%%% IMPORTS
\usepackage{hyperref}
\usepackage{amsmath,amssymb,amsfonts}
\usepackage{mathtools}						% Load after amsmath and mathspec. Needed for custom delimiters
\usepackage{algorithmic}
\usepackage{graphicx}
\usepackage{textcomp}
\usepackage{subcaption}
\usepackage[shortlabels]{enumitem}
\usepackage[ruled,vlined]{algorithm2e}
\usepackage{float}
\usepackage{wrapfig}
% \usepacakge{color}						
\usepackage{xcolor}
\usepackage{setspace}
\usepackage{wasysym}
\usepackage{xspace}							% Provides \xspace 

\usepackage{lipsum}  						% Remove this, it's for the template only
\usepackage{tabularx}  						% Used in the template. Nice package, but use with care. Remove by default 
\usepackage{etoolbox}						% Provides \iftoggle
% \usepackage[mathletters]{ucs}
\usepackage[utf8]{inputenc}
%%%%%%%%%%%%%%%%%%%%%%%%%%%%%%%%%%%%%%%%%%%%%%%%%%%%%%%%%%%%%%%%%%%%%%%%%%%%%%%%%%%%%%%%%%%%
%%%%%%%%%%%%%%%%%%%%%%%%%%%%%%%%%%%%%%%%%%%%%%%%%%%%%%% ENVIRONMENTS
\newtheorem{dfn}{Definition}
\newtheorem{asm}{Assumption}
\newtheorem{prp}{Proposition}
\newtheorem{thm}{Theorem}
\newtheorem{crl}{Corollary}
\newtheorem{rem}{Remark}
\newtheorem{exm}{Example}

%%%%%%%%%%%%%%%%%%%%%%%%%%%%%%%%%%%%%%%%%%%%%%%%%%%%%%%%%%%%%%%%%%%%%%%%%%%%%%%%%%%%%%%%%%%%
%%%%%%%%%%%%%%%%%%%%%%%%%%%%%%%%%%%%%%%%%%%%%%%%%%%%%%% BIBLIOGRAPHY SETUP
\usepackage[
	backend=biber,
	maxnames=5,
	style=alphabetic,
	sorting=ydnt,
	giveninits=true,
	  doi=false,
	  url=false,
	  isbn=false	
%	citestyle=verbose-ibid
	]{biblatex}
\addbibresource{../../bib/AIDA__Nov2023.bib}
\addbibresource{../../bib/Osinenko__Novp2023.bib}

%%%%%%%%%%%%%%%%%%%%%%%%%%%%%%%%%%%%%%%%%%%%%%%%%%%%%%%%%%%%%%%%%%%%%%%%%%%%%%%%%%%%%%%%%%%%
%%%%%%%%%%%%%%%%%%%%%%%%%%%%%%%%%%%%%%%%%%%%%%%%%%%%%%% PREAMBLE 

\newtoggle{mlnotation}	% True: use ML notation, false: use control theory notation

\toggletrue{mlnotation}
\iftoggle{mlnotation}
{
\input{../../preambleAIDA_ML__Nov2023}
}
{
\input{../../preambleAIDA_control__Nov2023}
}


\begin{document}

\section*{Problem statement}

Reinforcement learning deals with a special kind of Markov Decision Problems (MDPs) which have a normed vector space as a state-space:
\begin{equation}
\left(\states, \actions, \transit, \Cost \right),
\end{equation}
where:
\begin{enumerate}
\item $\states$ is the \textit{state space}, that is a normed vector space of all states of the given environment;
\item $\actions$ is the \textit{action space}, that is a set of all actions available to the agent;
\item $\transit : \states \times \actions \times \states \ \rightarrow \ \R$ is the \textit{transition probability density function} of the environment, that is such function that $\transit(\cdot \mid \state_{t}, \action_{t})$ is the probability density of the next state $s_{t + 1}$ conditioned on the current state $\state_{t}$ and current action $\action_{t}$;
\item $\Cost : \states \times \actions \rightarrow \mathbb{R}$ is the \textit{cost function} of the environment, that is a function that takes a state $\state_{t}$ and an action $\action_{t}$ and returns the immediate cost $\cost_{t}$ incurred upon the agent if it were to perform action $\action_{t}$ while in state $\state_{t}$;
%\item $\policies$ is the \textit{set of admissible policies}, that is a set of functions $\policy : \states \rightarrow \actions$, such that the distance between $s_t$ and $\G \subset \states$ tends to zero if $\action_{t} = \policy(\state_{t})$ (a set of \textit{stabilizing policies}). Note that it does not necessarilly include all such policies;
%\item $\G$ is the \textit{target set}, that is a bounded neighbourhood of the origin in $\states$.
\end{enumerate}
%Unlike regular Markov decision problems, MDP stabilization only admits stabilizing policies\ie such policies $\policy$ that the distance between $s_t$ and $\G$ tends to zero if $\action_{t} = \policy(\state_{t})$. For convenience we denote the set of all such admissible policies as $\policies_{\G}$.
\textbf{The goal of reinforcement learning} is to find a policy $\policy$ that minimizes $V^{\policy}(\state_0) = \E{\sum_{t = 0}^{\infty}\gamma^{t}\Cost(\state_{t}, \action_{t})}$ for some $\gamma \in (0, 1]$. The policy $\policy^{\ast}$ that solves this problem is commonly referred to as \textit{the optimal policy}.

\section*{Notation}

\begin{tabularx}{0.97\textwidth}{ |p{0.3\textwidth}|p{0.6\textwidth}| }  
    \hline
    $\policy$ & Policy, a law that generates actions from observations \\ 
    $\state, \State$ & State, as definite, resp., random value \\ 
    $\action, \Action$ & Action, as definite, resp., random value \\ 
    $\obs, \Obs$ & Observation, as definite, resp., random value \\ 
    $\states$ & State space \\
    $\actions$ & Action space \\
    $\obses$ & Observation space \\
    $\policies$ & Policy space \\
    $\transit$ & State dynamics law (function or probability distribution) \\ 
    $\Value^\policy$ & Total objective (value or cost) of policy $\policy$ \\
    $\Value^*$ & Optimum total objective (value or cost) under the optimal policy $\policy^*$ \\
    $\Advan^{\policy,\policy'}$ & Advantage of policy $\policy$ relative to policy $\policy'$ \\
    $\policy^\theta$ & Actor network with weights $\theta$ \\
    $\hat \Value^w$ & Critic network (state-valued) with weights $w$ \\
    \hline
\end{tabularx}

\end{document}